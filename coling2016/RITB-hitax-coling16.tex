%
% File coling2016.tex
%
% Contact: mutiyama@nict.go.jp
%%
%% Based on the style files for COLING-2014, which were, in turn,
%% Based on the style files for ACL-2014, which were, in turn,
%% Based on the style files for ACL-2013, which were, in turn,
%% Based on the style files for ACL-2012, which were, in turn,
%% based on the style files for ACL-2011, which were, in turn, 
%% based on the style files for ACL-2010, which were, in turn, 
%% based on the style files for ACL-IJCNLP-2009, which were, in turn,
%% based on the style files for EACL-2009 and IJCNLP-2008...

%% Based on the style files for EACL 2006 by 
%%e.agirre@ehu.es or Sergi.Balari@uab.es
%% and that of ACL 08 by Joakim Nivre and Noah Smith

\documentclass[11pt]{article}
\usepackage{coling2016}
\usepackage{times}
\usepackage{latexsym}


\usepackage{ amsmath }
\usepackage{ amssymb }
\usepackage{ amsthm }
\usepackage{ multicol }
\usepackage{ graphicx }
\usepackage{ subfig }
\usepackage{ enumerate }
\usepackage{ multirow }
\usepackage{ tabularx }
\usepackage{ setspace }
\usepackage{ rotating }
\usepackage{ wrapfig }
\usepackage{ fancybox }
\usepackage{ arydshln }
\usepackage{ caption }
\usepackage{ makecell }
\usepackage{ newclude }

% ========== coloring ==========
\usepackage[usenames,dvipsnames]{ color }
\usepackage{ xcolor }
\usepackage{ soul }

\definecolor{gray}{rgb}{0.4,0.4,0.4}
\definecolor{lightgray}{rgb}{0.8,0.8,0.8}
\definecolor{lightlightgray}{rgb}{0.9,0.9,0.9}
\definecolor{darkgray}{rgb}{0.2,0.2,0.2}
\definecolor{lightbrown}{rgb}{0.9,0.6,0.2}
\definecolor{brown}{rgb}{0.4,0.3,0.1}
\definecolor{orange}{rgb}{0.9,0.4,0}
\definecolor{DarkOrange}{HTML}{CC3300}

\usepackage{ url }
\usepackage{breakcites}
\usepackage[backref=page]{ hyperref }
\hypersetup{
	linktocpage,
    colorlinks=true,
    linkcolor=DarkOrange,
    citecolor=blue,
    urlcolor=Maroon
}

% creating a new command for \boldsymbol and \mathbf
\newcommand{\bs}[1]{\boldsymbol{#1}} % for symbols
\newcommand{\bm}[1]{\mathbf{#1}}     % for alphabets

%\setlength\titlebox{5cm}

% You can expand the titlebox if you need extra space
% to show all the authors. Please do not make the titlebox
% smaller than 5cm (the original size); we will check this
% in the camera-ready version and ask you to change it back.


\title{Large Scale Taxonomy Classification of Product Titles}

\author{First Authors \\
  Affiliation / Address line 1 \\
  Affiliation / Address line 2 \\
  Affiliation / Address line 3 \\
  {\tt email@domain} \\\And
  Second Author \\
  Affiliation / Address line 1 \\
  Affiliation / Address line 2 \\
  Affiliation / Address line 3 \\
  {\tt email@domain} \\}

\date{}

\begin{document}
\maketitle
\begin{abstract}
In this paper, we empirically evaluate classification of product listings into product taxonomies in an e-commerce setting.
The challenges in this task arise from fat to deep taxonomy sub-trees, high imbalance of listings residing in the leaf nodes of the branches; to the presence of noise at various levels depending upon data provenance.
We evaluate several classifiers on two different in-house product listing datasets from a major worldwide e-commerce organization -- 
one with minimal amount of noise and a taxonomy with a better balance of branches to listings at the leaf nodes and the other with a high level of noise and highly imbalanced number of listings in the leaf nodes of the taxonomy subtrees. 
We also evaluate the classifiers on an open source dataset which has been obtained by crawling Amazon with an available \textit{navigational taxonomy}. 
Our experiments show that both gradient boosted trees (GBTs) and convolutional neural network (CNN) with pretrained word embeddings yield the highest gains in error reduction on the in-house datasets while sparse linear classifiers and CNN outperforming GBTs on the Amazon dataset.

\end{abstract}

%%%%%%%%%%%%%%%%%%% Introduction
\section{Introduction}
\label{Sect:intro}
\include*{include/introduction}
%%%%%%%%%%%%%%%%%%% Experimental Setup
\section{Experimental Setup}
\label{Sect:experimental_setup}
\include*{include/experimental_setup}
%%%%%%%%%%%%%%%%%%% Related Work
\section{Related Work}
\label{Sect:related}
\include*{include/related}
%%%%%%%%%%%%%%%%%%% Results
\section{Results}
\label{Sect:results}
\include*{include/results}
%%%%%%%%%%%%%%%%%%% Acknowledgement
\section{Acknowledgements}
\label{Sect:aknowledgement}

BU-1, BU-2

%%%%%%%%%%%%%%%%%%% include your own bib file like this:
\bibliographystyle{acl}
\bibliography{RITB-hitax-coling16}

\end{document}
